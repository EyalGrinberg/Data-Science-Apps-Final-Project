% Options for packages loaded elsewhere
\PassOptionsToPackage{unicode}{hyperref}
\PassOptionsToPackage{hyphens}{url}
%
\documentclass[
]{article}
\usepackage{amsmath,amssymb}
\usepackage{iftex}
\ifPDFTeX
  \usepackage[T1]{fontenc}
  \usepackage[utf8]{inputenc}
  \usepackage{textcomp} % provide euro and other symbols
\else % if luatex or xetex
  \usepackage{unicode-math} % this also loads fontspec
  \defaultfontfeatures{Scale=MatchLowercase}
  \defaultfontfeatures[\rmfamily]{Ligatures=TeX,Scale=1}
\fi
\usepackage{lmodern}
\ifPDFTeX\else
  % xetex/luatex font selection
\fi
% Use upquote if available, for straight quotes in verbatim environments
\IfFileExists{upquote.sty}{\usepackage{upquote}}{}
\IfFileExists{microtype.sty}{% use microtype if available
  \usepackage[]{microtype}
  \UseMicrotypeSet[protrusion]{basicmath} % disable protrusion for tt fonts
}{}
\makeatletter
\@ifundefined{KOMAClassName}{% if non-KOMA class
  \IfFileExists{parskip.sty}{%
    \usepackage{parskip}
  }{% else
    \setlength{\parindent}{0pt}
    \setlength{\parskip}{6pt plus 2pt minus 1pt}}
}{% if KOMA class
  \KOMAoptions{parskip=half}}
\makeatother
\usepackage{xcolor}
\usepackage[margin=1in]{geometry}
\usepackage{color}
\usepackage{fancyvrb}
\newcommand{\VerbBar}{|}
\newcommand{\VERB}{\Verb[commandchars=\\\{\}]}
\DefineVerbatimEnvironment{Highlighting}{Verbatim}{commandchars=\\\{\}}
% Add ',fontsize=\small' for more characters per line
\usepackage{framed}
\definecolor{shadecolor}{RGB}{248,248,248}
\newenvironment{Shaded}{\begin{snugshade}}{\end{snugshade}}
\newcommand{\AlertTok}[1]{\textcolor[rgb]{0.94,0.16,0.16}{#1}}
\newcommand{\AnnotationTok}[1]{\textcolor[rgb]{0.56,0.35,0.01}{\textbf{\textit{#1}}}}
\newcommand{\AttributeTok}[1]{\textcolor[rgb]{0.13,0.29,0.53}{#1}}
\newcommand{\BaseNTok}[1]{\textcolor[rgb]{0.00,0.00,0.81}{#1}}
\newcommand{\BuiltInTok}[1]{#1}
\newcommand{\CharTok}[1]{\textcolor[rgb]{0.31,0.60,0.02}{#1}}
\newcommand{\CommentTok}[1]{\textcolor[rgb]{0.56,0.35,0.01}{\textit{#1}}}
\newcommand{\CommentVarTok}[1]{\textcolor[rgb]{0.56,0.35,0.01}{\textbf{\textit{#1}}}}
\newcommand{\ConstantTok}[1]{\textcolor[rgb]{0.56,0.35,0.01}{#1}}
\newcommand{\ControlFlowTok}[1]{\textcolor[rgb]{0.13,0.29,0.53}{\textbf{#1}}}
\newcommand{\DataTypeTok}[1]{\textcolor[rgb]{0.13,0.29,0.53}{#1}}
\newcommand{\DecValTok}[1]{\textcolor[rgb]{0.00,0.00,0.81}{#1}}
\newcommand{\DocumentationTok}[1]{\textcolor[rgb]{0.56,0.35,0.01}{\textbf{\textit{#1}}}}
\newcommand{\ErrorTok}[1]{\textcolor[rgb]{0.64,0.00,0.00}{\textbf{#1}}}
\newcommand{\ExtensionTok}[1]{#1}
\newcommand{\FloatTok}[1]{\textcolor[rgb]{0.00,0.00,0.81}{#1}}
\newcommand{\FunctionTok}[1]{\textcolor[rgb]{0.13,0.29,0.53}{\textbf{#1}}}
\newcommand{\ImportTok}[1]{#1}
\newcommand{\InformationTok}[1]{\textcolor[rgb]{0.56,0.35,0.01}{\textbf{\textit{#1}}}}
\newcommand{\KeywordTok}[1]{\textcolor[rgb]{0.13,0.29,0.53}{\textbf{#1}}}
\newcommand{\NormalTok}[1]{#1}
\newcommand{\OperatorTok}[1]{\textcolor[rgb]{0.81,0.36,0.00}{\textbf{#1}}}
\newcommand{\OtherTok}[1]{\textcolor[rgb]{0.56,0.35,0.01}{#1}}
\newcommand{\PreprocessorTok}[1]{\textcolor[rgb]{0.56,0.35,0.01}{\textit{#1}}}
\newcommand{\RegionMarkerTok}[1]{#1}
\newcommand{\SpecialCharTok}[1]{\textcolor[rgb]{0.81,0.36,0.00}{\textbf{#1}}}
\newcommand{\SpecialStringTok}[1]{\textcolor[rgb]{0.31,0.60,0.02}{#1}}
\newcommand{\StringTok}[1]{\textcolor[rgb]{0.31,0.60,0.02}{#1}}
\newcommand{\VariableTok}[1]{\textcolor[rgb]{0.00,0.00,0.00}{#1}}
\newcommand{\VerbatimStringTok}[1]{\textcolor[rgb]{0.31,0.60,0.02}{#1}}
\newcommand{\WarningTok}[1]{\textcolor[rgb]{0.56,0.35,0.01}{\textbf{\textit{#1}}}}
\usepackage{graphicx}
\makeatletter
\def\maxwidth{\ifdim\Gin@nat@width>\linewidth\linewidth\else\Gin@nat@width\fi}
\def\maxheight{\ifdim\Gin@nat@height>\textheight\textheight\else\Gin@nat@height\fi}
\makeatother
% Scale images if necessary, so that they will not overflow the page
% margins by default, and it is still possible to overwrite the defaults
% using explicit options in \includegraphics[width, height, ...]{}
\setkeys{Gin}{width=\maxwidth,height=\maxheight,keepaspectratio}
% Set default figure placement to htbp
\makeatletter
\def\fps@figure{htbp}
\makeatother
\setlength{\emergencystretch}{3em} % prevent overfull lines
\providecommand{\tightlist}{%
  \setlength{\itemsep}{0pt}\setlength{\parskip}{0pt}}
\setcounter{secnumdepth}{-\maxdimen} % remove section numbering
\ifLuaTeX
  \usepackage{selnolig}  % disable illegal ligatures
\fi
\IfFileExists{bookmark.sty}{\usepackage{bookmark}}{\usepackage{hyperref}}
\IfFileExists{xurl.sty}{\usepackage{xurl}}{} % add URL line breaks if available
\urlstyle{same}
\hypersetup{
  pdftitle={additional material for Data Exploratory},
  pdfauthor={Eyal Grinberg \& Yam Rozen},
  hidelinks,
  pdfcreator={LaTeX via pandoc}}

\title{additional material for Data Exploratory}
\author{Eyal Grinberg \& Yam Rozen}
\date{2023-07-19}

\begin{document}
\maketitle

\begin{Shaded}
\begin{Highlighting}[]
\NormalTok{knitr}\SpecialCharTok{::}\NormalTok{opts\_chunk}\SpecialCharTok{$}\FunctionTok{set}\NormalTok{(}\AttributeTok{fig.width=}\DecValTok{20}\NormalTok{, }\AttributeTok{fig.height=}\DecValTok{20}\NormalTok{)}
\end{Highlighting}
\end{Shaded}

packages and libraries

\begin{Shaded}
\begin{Highlighting}[]
\ControlFlowTok{if}\NormalTok{(F)\{}
  \FunctionTok{install.packages}\NormalTok{(}\FunctionTok{c}\NormalTok{(}\StringTok{"DataExplorer"}\NormalTok{,}\StringTok{"SmartEDA"}\NormalTok{,}\StringTok{"explore"}\NormalTok{,}\StringTok{"tidytext"}\NormalTok{,}\StringTok{"dataMaid"}\NormalTok{, }\StringTok{"explore"}\NormalTok{))}
\NormalTok{\}}
\end{Highlighting}
\end{Shaded}

\begin{Shaded}
\begin{Highlighting}[]
\FunctionTok{library}\NormalTok{(DataExplorer)}
\FunctionTok{library}\NormalTok{(SmartEDA)}
\end{Highlighting}
\end{Shaded}

\begin{verbatim}
## Registered S3 method overwritten by 'GGally':
##   method from   
##   +.gg   ggplot2
\end{verbatim}

\begin{Shaded}
\begin{Highlighting}[]
\FunctionTok{library}\NormalTok{(tidyverse)}
\end{Highlighting}
\end{Shaded}

\begin{verbatim}
## -- Attaching core tidyverse packages ------------------------ tidyverse 2.0.0 --
## v dplyr     1.1.1     v readr     2.1.4
## v forcats   1.0.0     v stringr   1.5.0
## v ggplot2   3.4.1     v tibble    3.2.1
## v lubridate 1.9.2     v tidyr     1.3.0
## v purrr     1.0.1
\end{verbatim}

\begin{verbatim}
## -- Conflicts ------------------------------------------ tidyverse_conflicts() --
## x dplyr::filter() masks stats::filter()
## x dplyr::lag()    masks stats::lag()
## i Use the ]8;;http://conflicted.r-lib.org/conflicted package]8;; to force all conflicts to become errors
\end{verbatim}

\begin{Shaded}
\begin{Highlighting}[]
\FunctionTok{library}\NormalTok{(dataMaid)}
\end{Highlighting}
\end{Shaded}

\begin{verbatim}
## 
## Attaching package: 'dataMaid'
## 
## The following object is masked from 'package:dplyr':
## 
##     summarize
\end{verbatim}

\begin{Shaded}
\begin{Highlighting}[]
\FunctionTok{library}\NormalTok{(dplyr)}
\FunctionTok{library}\NormalTok{(explore)}
\FunctionTok{library}\NormalTok{(ggplot2)}
\end{Highlighting}
\end{Shaded}

\begin{Shaded}
\begin{Highlighting}[]
\NormalTok{data\_food\_train }\OtherTok{\textless{}{-}} \FunctionTok{read\_csv}\NormalTok{(}\StringTok{"data/food\_train.csv"}\NormalTok{)}
\end{Highlighting}
\end{Shaded}

\begin{verbatim}
## Rows: 31751 Columns: 8
## -- Column specification --------------------------------------------------------
## Delimiter: ","
## chr (6): brand, description, ingredients, serving_size_unit, household_servi...
## dbl (2): idx, serving_size
## 
## i Use `spec()` to retrieve the full column specification for this data.
## i Specify the column types or set `show_col_types = FALSE` to quiet this message.
\end{verbatim}

\begin{Shaded}
\begin{Highlighting}[]
\FunctionTok{ExpReport}\NormalTok{(data\_food\_train, }\AttributeTok{Target =} \StringTok{"category"}\NormalTok{, }\AttributeTok{op\_file =} \StringTok{"Report.html"}\NormalTok{, }\AttributeTok{op\_dir =} \FunctionTok{getwd}\NormalTok{())}
\end{Highlighting}
\end{Shaded}

\begin{Shaded}
\begin{Highlighting}[]
\FunctionTok{makeDataReport}\NormalTok{(}\AttributeTok{data =}\NormalTok{ data\_food\_train, }\AttributeTok{output =} \StringTok{"html"}\NormalTok{, }\AttributeTok{replace =} \ConstantTok{TRUE}\NormalTok{)}
\end{Highlighting}
\end{Shaded}

\begin{Shaded}
\begin{Highlighting}[]
\FunctionTok{create\_report}\NormalTok{(}\AttributeTok{data =}\NormalTok{ data\_food\_train)}
\end{Highlighting}
\end{Shaded}

\begin{Shaded}
\begin{Highlighting}[]
\NormalTok{data\_nutrients }\OtherTok{\textless{}{-}} \FunctionTok{read\_csv}\NormalTok{(}\StringTok{"data/nutrients.csv"}\NormalTok{)}
\end{Highlighting}
\end{Shaded}

\begin{verbatim}
## Rows: 235 Columns: 3
## -- Column specification --------------------------------------------------------
## Delimiter: ","
## chr (2): name, unit_name
## dbl (1): nutrient_id
## 
## i Use `spec()` to retrieve the full column specification for this data.
## i Specify the column types or set `show_col_types = FALSE` to quiet this message.
\end{verbatim}

\begin{Shaded}
\begin{Highlighting}[]
\NormalTok{data\_food\_nutrients }\OtherTok{\textless{}{-}} \FunctionTok{read\_csv}\NormalTok{(}\StringTok{"data/food\_nutrients.csv"}\NormalTok{)}
\end{Highlighting}
\end{Shaded}

\begin{verbatim}
## Rows: 493054 Columns: 3
## -- Column specification --------------------------------------------------------
## Delimiter: ","
## dbl (3): idx, nutrient_id, amount
## 
## i Use `spec()` to retrieve the full column specification for this data.
## i Specify the column types or set `show_col_types = FALSE` to quiet this message.
\end{verbatim}

\begin{Shaded}
\begin{Highlighting}[]
\NormalTok{data\_food\_test }\OtherTok{\textless{}{-}} \FunctionTok{read\_csv}\NormalTok{(}\StringTok{"data/food\_test.csv"}\NormalTok{)}
\end{Highlighting}
\end{Shaded}

\begin{verbatim}
## Rows: 3525 Columns: 7
## -- Column specification --------------------------------------------------------
## Delimiter: ","
## chr (5): brand, description, ingredients, serving_size_unit, household_servi...
## dbl (2): idx, serving_size
## 
## i Use `spec()` to retrieve the full column specification for this data.
## i Specify the column types or set `show_col_types = FALSE` to quiet this message.
\end{verbatim}

\begin{Shaded}
\begin{Highlighting}[]
\NormalTok{merged\_df\_nutrients }\OtherTok{\textless{}{-}} \FunctionTok{merge}\NormalTok{(data\_food\_nutrients, data\_nutrients, }\AttributeTok{by =} \StringTok{"nutrient\_id"}\NormalTok{, }\AttributeTok{all.x =} \ConstantTok{TRUE}\NormalTok{)}
\NormalTok{merged\_df\_nutrients }\OtherTok{\textless{}{-}}\NormalTok{ merged\_df\_nutrients }\SpecialCharTok{\%\textgreater{}\%} \FunctionTok{arrange}\NormalTok{(idx)}
\end{Highlighting}
\end{Shaded}

\begin{Shaded}
\begin{Highlighting}[]
\FunctionTok{library}\NormalTok{(tidytext)}
\end{Highlighting}
\end{Shaded}

\begin{Shaded}
\begin{Highlighting}[]
\NormalTok{data\_train\_tokenized\_description }\OtherTok{\textless{}{-}}\NormalTok{ data\_food\_train }\SpecialCharTok{\%\textgreater{}\%} \FunctionTok{unnest\_tokens}\NormalTok{(word, description) }\SpecialCharTok{\%\textgreater{}\%} \FunctionTok{count}\NormalTok{(category, word, }\AttributeTok{sort =} \ConstantTok{TRUE}\NormalTok{)}
\NormalTok{data\_train\_tokenized\_description\_grouped\_by\_cat }\OtherTok{\textless{}{-}}\NormalTok{ data\_train\_tokenized\_description }\SpecialCharTok{\%\textgreater{}\%} \FunctionTok{group\_by}\NormalTok{(category) }\SpecialCharTok{\%\textgreater{}\%} \FunctionTok{summarise}\NormalTok{(word, n)}
\end{Highlighting}
\end{Shaded}

\begin{verbatim}
## Warning: Returning more (or less) than 1 row per `summarise()` group was deprecated in
## dplyr 1.1.0.
## i Please use `reframe()` instead.
## i When switching from `summarise()` to `reframe()`, remember that `reframe()`
##   always returns an ungrouped data frame and adjust accordingly.
## Call `lifecycle::last_lifecycle_warnings()` to see where this warning was
## generated.
\end{verbatim}

\begin{verbatim}
## `summarise()` has grouped output by 'category'. You can override using the
## `.groups` argument.
\end{verbatim}

\begin{Shaded}
\begin{Highlighting}[]
\NormalTok{data\_chocolate }\OtherTok{\textless{}{-}}\NormalTok{ data\_train\_tokenized\_description\_grouped\_by\_cat[data\_train\_tokenized\_description\_grouped\_by\_cat}\SpecialCharTok{$}\NormalTok{category }\SpecialCharTok{==} \StringTok{"chocolate"}\NormalTok{ , ]}
\NormalTok{data\_candy }\OtherTok{\textless{}{-}}\NormalTok{ data\_train\_tokenized\_description\_grouped\_by\_cat[data\_train\_tokenized\_description\_grouped\_by\_cat}\SpecialCharTok{$}\NormalTok{category }\SpecialCharTok{==} \StringTok{"candy"}\NormalTok{ , ]}
\NormalTok{data\_popcorn\_peanuts\_seeds\_related\_snacks }\OtherTok{\textless{}{-}}\NormalTok{ data\_train\_tokenized\_description\_grouped\_by\_cat[data\_train\_tokenized\_description\_grouped\_by\_cat}\SpecialCharTok{$}\NormalTok{category }\SpecialCharTok{==} \StringTok{"popcorn\_peanuts\_seeds\_related\_snacks"}\NormalTok{ , ]}

\CommentTok{\#data\_train\_tokenized\_description\_grouped\_by\_cat$word \textless{}{-} factor(data\_train\_tokenized\_description\_grouped\_by\_cat$word)}
\CommentTok{\#create\_report(data = data\_train\_tokenized\_description\_grouped\_by\_cat, y = "n")}
\CommentTok{\#ExpReport(data\_chocolate, Target = "word", op\_file = "Report.html", op\_dir = getwd())}
\end{Highlighting}
\end{Shaded}

\begin{Shaded}
\begin{Highlighting}[]
\NormalTok{data\_train\_tokenized\_description\_2\_tokens }\OtherTok{\textless{}{-}}\NormalTok{ data\_food\_train }\SpecialCharTok{\%\textgreater{}\%} \FunctionTok{unnest\_tokens}\NormalTok{(bigram, description, }\AttributeTok{token =} \StringTok{"ngrams"}\NormalTok{, }\AttributeTok{n =} \DecValTok{2}\NormalTok{) }\SpecialCharTok{\%\textgreater{}\%} \FunctionTok{count}\NormalTok{(category, bigram, }\AttributeTok{sort =} \ConstantTok{TRUE}\NormalTok{)}
\NormalTok{data\_train\_tokenized\_description\_grouped\_by\_cat\_2\_tokens }\OtherTok{\textless{}{-}}\NormalTok{ data\_train\_tokenized\_description\_2\_tokens }\SpecialCharTok{\%\textgreater{}\%} \FunctionTok{group\_by}\NormalTok{(category) }\SpecialCharTok{\%\textgreater{}\%} \FunctionTok{summarise}\NormalTok{(bigram, n)}
\end{Highlighting}
\end{Shaded}

\begin{verbatim}
## Warning: Returning more (or less) than 1 row per `summarise()` group was deprecated in
## dplyr 1.1.0.
## i Please use `reframe()` instead.
## i When switching from `summarise()` to `reframe()`, remember that `reframe()`
##   always returns an ungrouped data frame and adjust accordingly.
## Call `lifecycle::last_lifecycle_warnings()` to see where this warning was
## generated.
\end{verbatim}

\begin{verbatim}
## `summarise()` has grouped output by 'category'. You can override using the
## `.groups` argument.
\end{verbatim}

\begin{Shaded}
\begin{Highlighting}[]
\NormalTok{data\_popcorn\_2\_tokens }\OtherTok{\textless{}{-}}\NormalTok{ data\_train\_tokenized\_description\_grouped\_by\_cat\_2\_tokens[data\_train\_tokenized\_description\_grouped\_by\_cat\_2\_tokens}\SpecialCharTok{$}\NormalTok{category }\SpecialCharTok{==} \StringTok{"popcorn\_peanuts\_seeds\_related\_snacks"}\NormalTok{ , ]}
\NormalTok{data\_chocolate\_2\_tokens }\OtherTok{\textless{}{-}}\NormalTok{ data\_train\_tokenized\_description\_grouped\_by\_cat\_2\_tokens[data\_train\_tokenized\_description\_grouped\_by\_cat\_2\_tokens}\SpecialCharTok{$}\NormalTok{category }\SpecialCharTok{==} \StringTok{"chocolate"}\NormalTok{ , ]}
\end{Highlighting}
\end{Shaded}

\begin{Shaded}
\begin{Highlighting}[]
\NormalTok{data\_train\_tokenized\_ingredients }\OtherTok{\textless{}{-}}\NormalTok{ data\_food\_train }\SpecialCharTok{\%\textgreater{}\%} \FunctionTok{unnest\_tokens}\NormalTok{(word, ingredients) }\SpecialCharTok{\%\textgreater{}\%} \FunctionTok{count}\NormalTok{(category, word, }\AttributeTok{sort =} \ConstantTok{TRUE}\NormalTok{)}
\NormalTok{data\_train\_tokenized\_ingredients\_grouped\_by\_cat }\OtherTok{\textless{}{-}}\NormalTok{ data\_train\_tokenized\_ingredients }\SpecialCharTok{\%\textgreater{}\%} \FunctionTok{group\_by}\NormalTok{(category) }\SpecialCharTok{\%\textgreater{}\%} \FunctionTok{summarise}\NormalTok{(word, n)}
\end{Highlighting}
\end{Shaded}

\begin{verbatim}
## Warning: Returning more (or less) than 1 row per `summarise()` group was deprecated in
## dplyr 1.1.0.
## i Please use `reframe()` instead.
## i When switching from `summarise()` to `reframe()`, remember that `reframe()`
##   always returns an ungrouped data frame and adjust accordingly.
## Call `lifecycle::last_lifecycle_warnings()` to see where this warning was
## generated.
\end{verbatim}

\begin{verbatim}
## `summarise()` has grouped output by 'category'. You can override using the
## `.groups` argument.
\end{verbatim}

\begin{Shaded}
\begin{Highlighting}[]
\NormalTok{data\_chocolate\_ingrediends }\OtherTok{\textless{}{-}}\NormalTok{ data\_train\_tokenized\_ingredients\_grouped\_by\_cat[data\_train\_tokenized\_ingredients\_grouped\_by\_cat}\SpecialCharTok{$}\NormalTok{category }\SpecialCharTok{==} \StringTok{"chocolate"}\NormalTok{ , ]}
\NormalTok{data\_candy\_ingrediends }\OtherTok{\textless{}{-}}\NormalTok{ data\_train\_tokenized\_ingredients\_grouped\_by\_cat[data\_train\_tokenized\_ingredients\_grouped\_by\_cat}\SpecialCharTok{$}\NormalTok{category }\SpecialCharTok{==} \StringTok{"candy"}\NormalTok{ , ]}
\end{Highlighting}
\end{Shaded}

\begin{Shaded}
\begin{Highlighting}[]
\NormalTok{data\_train\_tokenized\_ingredients\_ngrams\_5 }\OtherTok{\textless{}{-}}\NormalTok{ data\_food\_train }\SpecialCharTok{\%\textgreater{}\%} \FunctionTok{unnest\_tokens}\NormalTok{(word, ingredients, }\AttributeTok{token =} \StringTok{"ngrams"}\NormalTok{, }\AttributeTok{n =} \DecValTok{5}\NormalTok{) }\SpecialCharTok{\%\textgreater{}\%} \FunctionTok{count}\NormalTok{(category, word, }\AttributeTok{sort =} \ConstantTok{TRUE}\NormalTok{)}
\NormalTok{data\_train\_tokenized\_ingredients\_grouped\_by\_cat\_ngrams\_5 }\OtherTok{\textless{}{-}}\NormalTok{ data\_train\_tokenized\_ingredients\_ngrams\_5 }\SpecialCharTok{\%\textgreater{}\%} \FunctionTok{group\_by}\NormalTok{(category) }\SpecialCharTok{\%\textgreater{}\%} \FunctionTok{summarise}\NormalTok{(word, n)}
\end{Highlighting}
\end{Shaded}

\begin{verbatim}
## Warning: Returning more (or less) than 1 row per `summarise()` group was deprecated in
## dplyr 1.1.0.
## i Please use `reframe()` instead.
## i When switching from `summarise()` to `reframe()`, remember that `reframe()`
##   always returns an ungrouped data frame and adjust accordingly.
## Call `lifecycle::last_lifecycle_warnings()` to see where this warning was
## generated.
\end{verbatim}

\begin{verbatim}
## `summarise()` has grouped output by 'category'. You can override using the
## `.groups` argument.
\end{verbatim}

\begin{Shaded}
\begin{Highlighting}[]
\NormalTok{merged\_df\_nutrients }\OtherTok{\textless{}{-}}\NormalTok{ merged\_df\_nutrients[}\SpecialCharTok{{-}}\DecValTok{1}\NormalTok{]}
\end{Highlighting}
\end{Shaded}

\begin{Shaded}
\begin{Highlighting}[]
\NormalTok{data\_food\_test}\SpecialCharTok{$}\NormalTok{category }\OtherTok{\textless{}{-}} \StringTok{"unknown"}
\NormalTok{data\_food }\OtherTok{\textless{}{-}} \FunctionTok{rbind}\NormalTok{(data\_food\_train, data\_food\_test)}
\NormalTok{merged\_df\_nutrients }\OtherTok{\textless{}{-}} \FunctionTok{merge}\NormalTok{(merged\_df\_nutrients, data\_food[}\FunctionTok{c}\NormalTok{(}\DecValTok{1}\NormalTok{,}\DecValTok{8}\NormalTok{)], }\AttributeTok{by =} \StringTok{"idx"}\NormalTok{, }\AttributeTok{all.x =} \ConstantTok{TRUE}\NormalTok{)}
\end{Highlighting}
\end{Shaded}

\begin{Shaded}
\begin{Highlighting}[]
\CommentTok{\# Grouping by \textquotesingle{}category\textquotesingle{} and \textquotesingle{}name\textquotesingle{}, then calculating the mean of \textquotesingle{}amount\textquotesingle{}}
\NormalTok{df\_nutrients\_mean\_by\_cat }\OtherTok{\textless{}{-}}\NormalTok{ merged\_df\_nutrients }\SpecialCharTok{\%\textgreater{}\%}
  \FunctionTok{group\_by}\NormalTok{(category, name) }\SpecialCharTok{\%\textgreater{}\%}
  \FunctionTok{mutate}\NormalTok{(}\AttributeTok{mean\_amount =} \FunctionTok{mean}\NormalTok{(amount))}
\end{Highlighting}
\end{Shaded}

\begin{Shaded}
\begin{Highlighting}[]
\NormalTok{nuts\_splitted\_by\_cat }\OtherTok{\textless{}{-}} \FunctionTok{split}\NormalTok{(merged\_df\_nutrients , merged\_df\_nutrients}\SpecialCharTok{$}\NormalTok{category)}
\CommentTok{\# creating a df for each category}
\NormalTok{cakes\_nutrients }\OtherTok{\textless{}{-}}\NormalTok{ nuts\_splitted\_by\_cat}\SpecialCharTok{$}\NormalTok{cakes\_cupcakes\_snack\_cakes}
\NormalTok{choco\_nutrients }\OtherTok{\textless{}{-}}\NormalTok{ nuts\_splitted\_by\_cat}\SpecialCharTok{$}\NormalTok{chocolate}
\NormalTok{popcorn\_nutrients }\OtherTok{\textless{}{-}}\NormalTok{ nuts\_splitted\_by\_cat}\SpecialCharTok{$}\NormalTok{popcorn\_peanuts\_seeds\_related\_snacks}
\NormalTok{candy\_nutrients }\OtherTok{\textless{}{-}}\NormalTok{ nuts\_splitted\_by\_cat}\SpecialCharTok{$}\NormalTok{candy}
\NormalTok{chips\_nutrients }\OtherTok{\textless{}{-}}\NormalTok{ nuts\_splitted\_by\_cat}\SpecialCharTok{$}\NormalTok{chips\_pretzels\_snacks}
\NormalTok{cookies\_nutrients }\OtherTok{\textless{}{-}}\NormalTok{ nuts\_splitted\_by\_cat}\SpecialCharTok{$}\NormalTok{cookies\_biscuits}
\NormalTok{test\_nutrients }\OtherTok{\textless{}{-}}\NormalTok{ nuts\_splitted\_by\_cat}\SpecialCharTok{$}\NormalTok{unknown}
\end{Highlighting}
\end{Shaded}

\begin{Shaded}
\begin{Highlighting}[]
\NormalTok{cakes\_mean\_by\_nut }\OtherTok{\textless{}{-}}\NormalTok{ cakes\_nutrients }\SpecialCharTok{\%\textgreater{}\%} \FunctionTok{group\_by}\NormalTok{(name) }\SpecialCharTok{\%\textgreater{}\%} \FunctionTok{reframe}\NormalTok{(}\AttributeTok{mean\_amount =} \FunctionTok{mean}\NormalTok{(amount)) }
\NormalTok{choco\_mean\_by\_nut }\OtherTok{\textless{}{-}}\NormalTok{ choco\_nutrients }\SpecialCharTok{\%\textgreater{}\%} \FunctionTok{group\_by}\NormalTok{(name) }\SpecialCharTok{\%\textgreater{}\%} \FunctionTok{reframe}\NormalTok{(}\AttributeTok{mean\_amount =} \FunctionTok{mean}\NormalTok{(amount))}
\NormalTok{popcorn\_mean\_by\_nut }\OtherTok{\textless{}{-}}\NormalTok{ popcorn\_nutrients }\SpecialCharTok{\%\textgreater{}\%} \FunctionTok{group\_by}\NormalTok{(name) }\SpecialCharTok{\%\textgreater{}\%} \FunctionTok{reframe}\NormalTok{(}\AttributeTok{mean\_amount =} \FunctionTok{mean}\NormalTok{(amount))}
\NormalTok{candy\_mean\_by\_nut }\OtherTok{\textless{}{-}}\NormalTok{ candy\_nutrients }\SpecialCharTok{\%\textgreater{}\%} \FunctionTok{group\_by}\NormalTok{(name) }\SpecialCharTok{\%\textgreater{}\%} \FunctionTok{reframe}\NormalTok{(}\AttributeTok{mean\_amount =} \FunctionTok{mean}\NormalTok{(amount))}
\NormalTok{chips\_mean\_by\_nut }\OtherTok{\textless{}{-}}\NormalTok{ chips\_nutrients }\SpecialCharTok{\%\textgreater{}\%} \FunctionTok{group\_by}\NormalTok{(name) }\SpecialCharTok{\%\textgreater{}\%} \FunctionTok{reframe}\NormalTok{(}\AttributeTok{mean\_amount =} \FunctionTok{mean}\NormalTok{(amount))}
\NormalTok{cookies\_mean\_by\_nut }\OtherTok{\textless{}{-}}\NormalTok{ cookies\_nutrients }\SpecialCharTok{\%\textgreater{}\%} \FunctionTok{group\_by}\NormalTok{(name) }\SpecialCharTok{\%\textgreater{}\%} \FunctionTok{reframe}\NormalTok{(}\AttributeTok{mean\_amount =} \FunctionTok{mean}\NormalTok{(amount))}
\NormalTok{test\_mean\_by\_nut }\OtherTok{\textless{}{-}}\NormalTok{ test\_nutrients }\SpecialCharTok{\%\textgreater{}\%} \FunctionTok{group\_by}\NormalTok{(name) }\SpecialCharTok{\%\textgreater{}\%} \FunctionTok{reframe}\NormalTok{(}\AttributeTok{mean\_amount =} \FunctionTok{mean}\NormalTok{(amount))}
\end{Highlighting}
\end{Shaded}

\begin{Shaded}
\begin{Highlighting}[]
\NormalTok{cakes\_num\_nutrients }\OtherTok{\textless{}{-}}\NormalTok{ cakes\_nutrients }\SpecialCharTok{\%\textgreater{}\%} \FunctionTok{group\_by}\NormalTok{(idx) }\SpecialCharTok{\%\textgreater{}\%} \FunctionTok{reframe}\NormalTok{(}\AttributeTok{num\_nutrients =} \FunctionTok{length}\NormalTok{(name))}
\NormalTok{choco\_num\_nutrients }\OtherTok{\textless{}{-}}\NormalTok{ choco\_nutrients }\SpecialCharTok{\%\textgreater{}\%} \FunctionTok{group\_by}\NormalTok{(idx) }\SpecialCharTok{\%\textgreater{}\%} \FunctionTok{reframe}\NormalTok{(}\AttributeTok{num\_nutrients =} \FunctionTok{length}\NormalTok{(name))}
\NormalTok{popcorn\_num\_nutrients }\OtherTok{\textless{}{-}}\NormalTok{ popcorn\_nutrients }\SpecialCharTok{\%\textgreater{}\%} \FunctionTok{group\_by}\NormalTok{(idx) }\SpecialCharTok{\%\textgreater{}\%} \FunctionTok{reframe}\NormalTok{(}\AttributeTok{num\_nutrients =} \FunctionTok{length}\NormalTok{(name))}
\NormalTok{candy\_num\_nutrients }\OtherTok{\textless{}{-}}\NormalTok{ candy\_nutrients }\SpecialCharTok{\%\textgreater{}\%} \FunctionTok{group\_by}\NormalTok{(idx) }\SpecialCharTok{\%\textgreater{}\%} \FunctionTok{reframe}\NormalTok{(}\AttributeTok{num\_nutrients =} \FunctionTok{length}\NormalTok{(name))}
\NormalTok{chips\_num\_nutrients }\OtherTok{\textless{}{-}}\NormalTok{ chips\_nutrients }\SpecialCharTok{\%\textgreater{}\%} \FunctionTok{group\_by}\NormalTok{(idx) }\SpecialCharTok{\%\textgreater{}\%} \FunctionTok{reframe}\NormalTok{(}\AttributeTok{num\_nutrients =} \FunctionTok{length}\NormalTok{(name))}
\NormalTok{cookies\_num\_nutrients }\OtherTok{\textless{}{-}}\NormalTok{ cookies\_nutrients }\SpecialCharTok{\%\textgreater{}\%} \FunctionTok{group\_by}\NormalTok{(idx) }\SpecialCharTok{\%\textgreater{}\%} \FunctionTok{reframe}\NormalTok{(}\AttributeTok{num\_nutrients =} \FunctionTok{length}\NormalTok{(name))}
\NormalTok{test\_num\_nutrients }\OtherTok{\textless{}{-}}\NormalTok{ test\_nutrients }\SpecialCharTok{\%\textgreater{}\%} \FunctionTok{group\_by}\NormalTok{(idx) }\SpecialCharTok{\%\textgreater{}\%} \FunctionTok{reframe}\NormalTok{(}\AttributeTok{num\_nutrients =} \FunctionTok{length}\NormalTok{(name))}
\end{Highlighting}
\end{Shaded}

\begin{Shaded}
\begin{Highlighting}[]
\NormalTok{plots\_func\_num\_nutriens\_by\_cat }\OtherTok{\textless{}{-}} \ControlFlowTok{function}\NormalTok{(x, title\_name) \{}
  \FunctionTok{hist}\NormalTok{(x}\SpecialCharTok{$}\NormalTok{num\_nutrients, }\AttributeTok{main =}\NormalTok{ title\_name)}
\NormalTok{\}}
\end{Highlighting}
\end{Shaded}

\begin{Shaded}
\begin{Highlighting}[]
\FunctionTok{plots\_func\_num\_nutriens\_by\_cat}\NormalTok{(cakes\_num\_nutrients, }\StringTok{"cakes\_num\_nutrients"}\NormalTok{)}
\end{Highlighting}
\end{Shaded}

\includegraphics{additional_material_for_Data_Exploratory_files/figure-latex/unnamed-chunk-24-1.pdf}

\begin{Shaded}
\begin{Highlighting}[]
\FunctionTok{plots\_func\_num\_nutriens\_by\_cat}\NormalTok{(choco\_num\_nutrients, }\StringTok{"choco\_num\_nutrients"}\NormalTok{)}
\end{Highlighting}
\end{Shaded}

\includegraphics{additional_material_for_Data_Exploratory_files/figure-latex/unnamed-chunk-24-2.pdf}

\begin{Shaded}
\begin{Highlighting}[]
\FunctionTok{plots\_func\_num\_nutriens\_by\_cat}\NormalTok{(popcorn\_num\_nutrients, }\StringTok{"popcorn\_num\_nutrients"}\NormalTok{)}
\end{Highlighting}
\end{Shaded}

\includegraphics{additional_material_for_Data_Exploratory_files/figure-latex/unnamed-chunk-24-3.pdf}

\begin{Shaded}
\begin{Highlighting}[]
\FunctionTok{plots\_func\_num\_nutriens\_by\_cat}\NormalTok{(candy\_num\_nutrients, }\StringTok{"candy\_num\_nutrients"}\NormalTok{)}
\end{Highlighting}
\end{Shaded}

\includegraphics{additional_material_for_Data_Exploratory_files/figure-latex/unnamed-chunk-24-4.pdf}

\begin{Shaded}
\begin{Highlighting}[]
\FunctionTok{plots\_func\_num\_nutriens\_by\_cat}\NormalTok{(chips\_num\_nutrients, }\StringTok{"chips\_num\_nutrients"}\NormalTok{)}
\end{Highlighting}
\end{Shaded}

\includegraphics{additional_material_for_Data_Exploratory_files/figure-latex/unnamed-chunk-24-5.pdf}

\begin{Shaded}
\begin{Highlighting}[]
\FunctionTok{plots\_func\_num\_nutriens\_by\_cat}\NormalTok{(cookies\_num\_nutrients, }\StringTok{"cookies\_num\_nutrients"}\NormalTok{)}
\end{Highlighting}
\end{Shaded}

\includegraphics{additional_material_for_Data_Exploratory_files/figure-latex/unnamed-chunk-24-6.pdf}

\begin{Shaded}
\begin{Highlighting}[]
\FunctionTok{plots\_func\_num\_nutriens\_by\_cat}\NormalTok{(test\_num\_nutrients, }\StringTok{"test\_num\_nutrients"}\NormalTok{)}
\end{Highlighting}
\end{Shaded}

\includegraphics{additional_material_for_Data_Exploratory_files/figure-latex/unnamed-chunk-24-7.pdf}

\begin{Shaded}
\begin{Highlighting}[]
\CommentTok{\# creating boxplots}
\FunctionTok{boxplot}\NormalTok{(cakes\_num\_nutrients}\SpecialCharTok{$}\NormalTok{num\_nutrients, choco\_num\_nutrients}\SpecialCharTok{$}\NormalTok{num\_nutrients, popcorn\_num\_nutrients}\SpecialCharTok{$}\NormalTok{num\_nutrients, candy\_num\_nutrients}\SpecialCharTok{$}\NormalTok{num\_nutrients, chips\_num\_nutrients}\SpecialCharTok{$}\NormalTok{num\_nutrients, cookies\_num\_nutrients}\SpecialCharTok{$}\NormalTok{num\_nutrients, test\_num\_nutrients}\SpecialCharTok{$}\NormalTok{num\_nutrients, }\AttributeTok{main =} \StringTok{"boxplot for num nutrients per category"}\NormalTok{, }\AttributeTok{names =} \FunctionTok{c}\NormalTok{(}\StringTok{"cakes"}\NormalTok{, }\StringTok{"chocolate"}\NormalTok{, }\StringTok{"popcorn"}\NormalTok{, }\StringTok{"candy"}\NormalTok{, }\StringTok{"chips"}\NormalTok{, }\StringTok{"cookies"}\NormalTok{, }\StringTok{"test"}\NormalTok{))}
\end{Highlighting}
\end{Shaded}

\includegraphics{additional_material_for_Data_Exploratory_files/figure-latex/unnamed-chunk-25-1.pdf}

\begin{Shaded}
\begin{Highlighting}[]
\CommentTok{\# create a df of the nutrients }
\NormalTok{m }\OtherTok{\textless{}{-}} \FunctionTok{merge}\NormalTok{(cakes\_mean\_by\_nut,candy\_mean\_by\_nut, }\AttributeTok{by =} \StringTok{"name"}\NormalTok{, }\AttributeTok{all =} \ConstantTok{TRUE}\NormalTok{)}
\NormalTok{m }\OtherTok{\textless{}{-}} \FunctionTok{merge}\NormalTok{(m,popcorn\_mean\_by\_nut, }\AttributeTok{by =} \StringTok{"name"}\NormalTok{, }\AttributeTok{all =} \ConstantTok{TRUE}\NormalTok{)}
\NormalTok{m }\OtherTok{\textless{}{-}} \FunctionTok{merge}\NormalTok{(m,choco\_mean\_by\_nut, }\AttributeTok{by =} \StringTok{"name"}\NormalTok{, }\AttributeTok{all =} \ConstantTok{TRUE}\NormalTok{)}
\end{Highlighting}
\end{Shaded}

\begin{verbatim}
## Warning in merge.data.frame(m, choco_mean_by_nut, by = "name", all = TRUE):
## column names 'mean_amount.x', 'mean_amount.y' are duplicated in the result
\end{verbatim}

\begin{Shaded}
\begin{Highlighting}[]
\NormalTok{m }\OtherTok{\textless{}{-}} \FunctionTok{merge}\NormalTok{(m,chips\_mean\_by\_nut, }\AttributeTok{by =} \StringTok{"name"}\NormalTok{, }\AttributeTok{all =} \ConstantTok{TRUE}\NormalTok{)}
\end{Highlighting}
\end{Shaded}

\begin{verbatim}
## Warning in merge.data.frame(m, chips_mean_by_nut, by = "name", all = TRUE):
## column names 'mean_amount.x', 'mean_amount.y' are duplicated in the result
\end{verbatim}

\begin{Shaded}
\begin{Highlighting}[]
\NormalTok{m }\OtherTok{\textless{}{-}} \FunctionTok{merge}\NormalTok{(m,cookies\_mean\_by\_nut, }\AttributeTok{by =} \StringTok{"name"}\NormalTok{, }\AttributeTok{all =} \ConstantTok{TRUE}\NormalTok{)}
\end{Highlighting}
\end{Shaded}

\begin{verbatim}
## Warning in merge.data.frame(m, cookies_mean_by_nut, by = "name", all = TRUE):
## column names 'mean_amount.x', 'mean_amount.y', 'mean_amount.x', 'mean_amount.y'
## are duplicated in the result
\end{verbatim}

\begin{Shaded}
\begin{Highlighting}[]
\NormalTok{nutrients\_mean\_amount\_with\_zero\_amounts }\OtherTok{\textless{}{-}} \FunctionTok{merge}\NormalTok{(m,test\_mean\_by\_nut, }\AttributeTok{by =} \StringTok{"name"}\NormalTok{, }\AttributeTok{all =} \ConstantTok{TRUE}\NormalTok{) }
\end{Highlighting}
\end{Shaded}

\begin{verbatim}
## Warning in merge.data.frame(m, test_mean_by_nut, by = "name", all = TRUE):
## column names 'mean_amount.x', 'mean_amount.y', 'mean_amount.x', 'mean_amount.y'
## are duplicated in the result
\end{verbatim}

\begin{Shaded}
\begin{Highlighting}[]
\FunctionTok{colnames}\NormalTok{(nutrients\_mean\_amount\_with\_zero\_amounts) }\OtherTok{\textless{}{-}} \FunctionTok{c}\NormalTok{(}\StringTok{"nutrient"}\NormalTok{, }\StringTok{"cakes"}\NormalTok{, }\StringTok{"candy"}\NormalTok{, }\StringTok{"popcorn"}\NormalTok{, }\StringTok{"chocolate"}\NormalTok{, }\StringTok{"chips"}\NormalTok{, }\StringTok{"cookies"}\NormalTok{, }\StringTok{"test"}\NormalTok{)}
\CommentTok{\# replace NAs with 0}
\NormalTok{nutrients\_mean\_amount\_with\_zero\_amounts[}\FunctionTok{is.na}\NormalTok{(nutrients\_mean\_amount\_with\_zero\_amounts)] }\OtherTok{\textless{}{-}} \DecValTok{0}  
\FunctionTok{rownames}\NormalTok{(nutrients\_mean\_amount\_with\_zero\_amounts) }\OtherTok{\textless{}{-}}\NormalTok{ nutrients\_mean\_amount\_with\_zero\_amounts[,}\DecValTok{1}\NormalTok{]}
\CommentTok{\#nutrients\_mean\_amount\_with\_zero\_amounts \textless{}{-} nutrients\_mean\_amount\_with\_zero\_amounts[,{-}1]}
\end{Highlighting}
\end{Shaded}

\begin{Shaded}
\begin{Highlighting}[]
\NormalTok{nuts\_pivoted }\OtherTok{\textless{}{-}}\NormalTok{ nutrients\_mean\_amount\_with\_zero\_amounts }\SpecialCharTok{\%\textgreater{}\%} \FunctionTok{pivot\_longer}\NormalTok{(}\SpecialCharTok{!}\NormalTok{nutrient, }\AttributeTok{names\_to =} \StringTok{"category"}\NormalTok{, }\AttributeTok{values\_to =} \StringTok{"mean\_amount"}\NormalTok{)}
\FunctionTok{ggplot}\NormalTok{(}\AttributeTok{data =}\NormalTok{ nuts\_pivoted, }\AttributeTok{mapping =} \FunctionTok{aes}\NormalTok{(}\AttributeTok{x =}\NormalTok{ category, }\AttributeTok{y =}\NormalTok{ mean\_amount, }\AttributeTok{color =}\NormalTok{ category)) }\SpecialCharTok{+}
  \FunctionTok{geom\_point}\NormalTok{() }\SpecialCharTok{+}
  \FunctionTok{facet\_wrap}\NormalTok{(. }\SpecialCharTok{\textasciitilde{}}\NormalTok{ nutrient, }\AttributeTok{scales =} \StringTok{"free\_y"}\NormalTok{) }\SpecialCharTok{+}
  \FunctionTok{labs}\NormalTok{(}\AttributeTok{title =} \StringTok{"category vs. nutrient mean amount of each nutrient"}\NormalTok{,}
       \AttributeTok{subtitle =} \StringTok{"Faceted by nutrient"}\NormalTok{,}
       \AttributeTok{x =} \StringTok{"category"}\NormalTok{, }\AttributeTok{y =}\StringTok{"mean\_amount"}\NormalTok{) }\SpecialCharTok{+} 
  \FunctionTok{theme}\NormalTok{(}\AttributeTok{strip.text.x =} \FunctionTok{element\_text}\NormalTok{(}\AttributeTok{size =} \DecValTok{4}\NormalTok{, }\AttributeTok{margin =} \FunctionTok{margin}\NormalTok{()),}
        \AttributeTok{strip.text.y =} \FunctionTok{element\_text}\NormalTok{(}\AttributeTok{size =} \DecValTok{20}\NormalTok{, }\AttributeTok{margin =} \FunctionTok{margin}\NormalTok{()))}
\end{Highlighting}
\end{Shaded}

\includegraphics{additional_material_for_Data_Exploratory_files/figure-latex/unnamed-chunk-27-1.pdf}

\begin{Shaded}
\begin{Highlighting}[]
\CommentTok{\#ggsave("plot.png", plot, width = 10, height = 10, units = c("in"))}
\end{Highlighting}
\end{Shaded}

\begin{Shaded}
\begin{Highlighting}[]
\CommentTok{\#transpoze the df}
\NormalTok{transpozed\_mean\_amount\_with\_zero\_amounts }\OtherTok{\textless{}{-}} \FunctionTok{as.data.frame}\NormalTok{( }\FunctionTok{t}\NormalTok{(nutrients\_mean\_amount\_with\_zero\_amounts) )}
\FunctionTok{colnames}\NormalTok{(transpozed\_mean\_amount\_with\_zero\_amounts) }\OtherTok{\textless{}{-}}\NormalTok{ transpozed\_mean\_amount\_with\_zero\_amounts[}\DecValTok{1}\NormalTok{,]}
\NormalTok{transpozed\_mean\_amount\_with\_zero\_amounts }\OtherTok{\textless{}{-}}\NormalTok{ transpozed\_mean\_amount\_with\_zero\_amounts[}\SpecialCharTok{{-}}\DecValTok{1}\NormalTok{,]}
\NormalTok{transpozed\_mean\_amount\_with\_zero\_amounts }\OtherTok{\textless{}{-}} \FunctionTok{as.data.frame}\NormalTok{(}\FunctionTok{sapply}\NormalTok{(transpozed\_mean\_amount\_with\_zero\_amounts, as.numeric))}
\FunctionTok{rownames}\NormalTok{(transpozed\_mean\_amount\_with\_zero\_amounts) }\OtherTok{\textless{}{-}} \FunctionTok{c}\NormalTok{(}\StringTok{"cakes"}\NormalTok{, }\StringTok{"candy"}\NormalTok{, }\StringTok{"popcorn"}\NormalTok{, }\StringTok{"chocolate"}\NormalTok{, }\StringTok{"chips"}\NormalTok{, }\StringTok{"cookies"}\NormalTok{, }\StringTok{"test"}\NormalTok{)}
\end{Highlighting}
\end{Shaded}


\end{document}
